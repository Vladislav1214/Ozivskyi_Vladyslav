\documentclass[11pt]{article}

    \usepackage[breakable]{tcolorbox}
    \usepackage{parskip} % Stop auto-indenting (to mimic markdown behaviour)
    

    % Basic figure setup, for now with no caption control since it's done
    % automatically by Pandoc (which extracts ![](path) syntax from Markdown).
    \usepackage{graphicx}
    % Keep aspect ratio if custom image width or height is specified
    \setkeys{Gin}{keepaspectratio}
    % Maintain compatibility with old templates. Remove in nbconvert 6.0
    \let\Oldincludegraphics\includegraphics
    % Ensure that by default, figures have no caption (until we provide a
    % proper Figure object with a Caption API and a way to capture that
    % in the conversion process - todo).
    \usepackage{caption}
    \DeclareCaptionFormat{nocaption}{}
    \captionsetup{format=nocaption,aboveskip=0pt,belowskip=0pt}

    \usepackage{float}
    \floatplacement{figure}{H} % forces figures to be placed at the correct location
    \usepackage{xcolor} % Allow colors to be defined
    \usepackage{enumerate} % Needed for markdown enumerations to work
    \usepackage{geometry} % Used to adjust the document margins
    \usepackage{amsmath} % Equations
    \usepackage{amssymb} % Equations
    \usepackage{textcomp} % defines textquotesingle
    % Hack from http://tex.stackexchange.com/a/47451/13684:
    \AtBeginDocument{%
        \def\PYZsq{\textquotesingle}% Upright quotes in Pygmentized code
    }
    \usepackage{upquote} % Upright quotes for verbatim code
    \usepackage{eurosym} % defines \euro

    \usepackage{iftex}
    \ifPDFTeX
        \usepackage[T1]{fontenc}
        \IfFileExists{alphabeta.sty}{
              \usepackage{alphabeta}
          }{
              \usepackage[mathletters]{ucs}
              \usepackage[utf8x]{inputenc}
          }
    \else
        \usepackage{fontspec}
        \usepackage{unicode-math}
    \fi

    \usepackage{fancyvrb} % verbatim replacement that allows latex
    \usepackage{grffile} % extends the file name processing of package graphics
                         % to support a larger range
    \makeatletter % fix for old versions of grffile with XeLaTeX
    \@ifpackagelater{grffile}{2019/11/01}
    {
      % Do nothing on new versions
    }
    {
      \def\Gread@@xetex#1{%
        \IfFileExists{"\Gin@base".bb}%
        {\Gread@eps{\Gin@base.bb}}%
        {\Gread@@xetex@aux#1}%
      }
    }
    \makeatother
    \usepackage[Export]{adjustbox} % Used to constrain images to a maximum size
    \adjustboxset{max size={0.9\linewidth}{0.9\paperheight}}

    % The hyperref package gives us a pdf with properly built
    % internal navigation ('pdf bookmarks' for the table of contents,
    % internal cross-reference links, web links for URLs, etc.)
    \usepackage{hyperref}
    % The default LaTeX title has an obnoxious amount of whitespace. By default,
    % titling removes some of it. It also provides customization options.
    \usepackage{titling}
    \usepackage{longtable} % longtable support required by pandoc >1.10
    \usepackage{booktabs}  % table support for pandoc > 1.12.2
    \usepackage{array}     % table support for pandoc >= 2.11.3
    \usepackage{calc}      % table minipage width calculation for pandoc >= 2.11.1
    \usepackage[inline]{enumitem} % IRkernel/repr support (it uses the enumerate* environment)
    \usepackage[normalem]{ulem} % ulem is needed to support strikethroughs (\sout)
                                % normalem makes italics be italics, not underlines
    \usepackage{soul}      % strikethrough (\st) support for pandoc >= 3.0.0
    \usepackage{mathrsfs}
    

    
    % Colors for the hyperref package
    \definecolor{urlcolor}{rgb}{0,.145,.698}
    \definecolor{linkcolor}{rgb}{.71,0.21,0.01}
    \definecolor{citecolor}{rgb}{.12,.54,.11}

    % ANSI colors
    \definecolor{ansi-black}{HTML}{3E424D}
    \definecolor{ansi-black-intense}{HTML}{282C36}
    \definecolor{ansi-red}{HTML}{E75C58}
    \definecolor{ansi-red-intense}{HTML}{B22B31}
    \definecolor{ansi-green}{HTML}{00A250}
    \definecolor{ansi-green-intense}{HTML}{007427}
    \definecolor{ansi-yellow}{HTML}{DDB62B}
    \definecolor{ansi-yellow-intense}{HTML}{B27D12}
    \definecolor{ansi-blue}{HTML}{208FFB}
    \definecolor{ansi-blue-intense}{HTML}{0065CA}
    \definecolor{ansi-magenta}{HTML}{D160C4}
    \definecolor{ansi-magenta-intense}{HTML}{A03196}
    \definecolor{ansi-cyan}{HTML}{60C6C8}
    \definecolor{ansi-cyan-intense}{HTML}{258F8F}
    \definecolor{ansi-white}{HTML}{C5C1B4}
    \definecolor{ansi-white-intense}{HTML}{A1A6B2}
    \definecolor{ansi-default-inverse-fg}{HTML}{FFFFFF}
    \definecolor{ansi-default-inverse-bg}{HTML}{000000}

    % common color for the border for error outputs.
    \definecolor{outerrorbackground}{HTML}{FFDFDF}

    % commands and environments needed by pandoc snippets
    % extracted from the output of `pandoc -s`
    \providecommand{\tightlist}{%
      \setlength{\itemsep}{0pt}\setlength{\parskip}{0pt}}
    \DefineVerbatimEnvironment{Highlighting}{Verbatim}{commandchars=\\\{\}}
    % Add ',fontsize=\small' for more characters per line
    \newenvironment{Shaded}{}{}
    \newcommand{\KeywordTok}[1]{\textcolor[rgb]{0.00,0.44,0.13}{\textbf{{#1}}}}
    \newcommand{\DataTypeTok}[1]{\textcolor[rgb]{0.56,0.13,0.00}{{#1}}}
    \newcommand{\DecValTok}[1]{\textcolor[rgb]{0.25,0.63,0.44}{{#1}}}
    \newcommand{\BaseNTok}[1]{\textcolor[rgb]{0.25,0.63,0.44}{{#1}}}
    \newcommand{\FloatTok}[1]{\textcolor[rgb]{0.25,0.63,0.44}{{#1}}}
    \newcommand{\CharTok}[1]{\textcolor[rgb]{0.25,0.44,0.63}{{#1}}}
    \newcommand{\StringTok}[1]{\textcolor[rgb]{0.25,0.44,0.63}{{#1}}}
    \newcommand{\CommentTok}[1]{\textcolor[rgb]{0.38,0.63,0.69}{\textit{{#1}}}}
    \newcommand{\OtherTok}[1]{\textcolor[rgb]{0.00,0.44,0.13}{{#1}}}
    \newcommand{\AlertTok}[1]{\textcolor[rgb]{1.00,0.00,0.00}{\textbf{{#1}}}}
    \newcommand{\FunctionTok}[1]{\textcolor[rgb]{0.02,0.16,0.49}{{#1}}}
    \newcommand{\RegionMarkerTok}[1]{{#1}}
    \newcommand{\ErrorTok}[1]{\textcolor[rgb]{1.00,0.00,0.00}{\textbf{{#1}}}}
    \newcommand{\NormalTok}[1]{{#1}}

    % Additional commands for more recent versions of Pandoc
    \newcommand{\ConstantTok}[1]{\textcolor[rgb]{0.53,0.00,0.00}{{#1}}}
    \newcommand{\SpecialCharTok}[1]{\textcolor[rgb]{0.25,0.44,0.63}{{#1}}}
    \newcommand{\VerbatimStringTok}[1]{\textcolor[rgb]{0.25,0.44,0.63}{{#1}}}
    \newcommand{\SpecialStringTok}[1]{\textcolor[rgb]{0.73,0.40,0.53}{{#1}}}
    \newcommand{\ImportTok}[1]{{#1}}
    \newcommand{\DocumentationTok}[1]{\textcolor[rgb]{0.73,0.13,0.13}{\textit{{#1}}}}
    \newcommand{\AnnotationTok}[1]{\textcolor[rgb]{0.38,0.63,0.69}{\textbf{\textit{{#1}}}}}
    \newcommand{\CommentVarTok}[1]{\textcolor[rgb]{0.38,0.63,0.69}{\textbf{\textit{{#1}}}}}
    \newcommand{\VariableTok}[1]{\textcolor[rgb]{0.10,0.09,0.49}{{#1}}}
    \newcommand{\ControlFlowTok}[1]{\textcolor[rgb]{0.00,0.44,0.13}{\textbf{{#1}}}}
    \newcommand{\OperatorTok}[1]{\textcolor[rgb]{0.40,0.40,0.40}{{#1}}}
    \newcommand{\BuiltInTok}[1]{{#1}}
    \newcommand{\ExtensionTok}[1]{{#1}}
    \newcommand{\PreprocessorTok}[1]{\textcolor[rgb]{0.74,0.48,0.00}{{#1}}}
    \newcommand{\AttributeTok}[1]{\textcolor[rgb]{0.49,0.56,0.16}{{#1}}}
    \newcommand{\InformationTok}[1]{\textcolor[rgb]{0.38,0.63,0.69}{\textbf{\textit{{#1}}}}}
    \newcommand{\WarningTok}[1]{\textcolor[rgb]{0.38,0.63,0.69}{\textbf{\textit{{#1}}}}}
    \makeatletter
    \newsavebox\pandoc@box
    \newcommand*\pandocbounded[1]{%
      \sbox\pandoc@box{#1}%
      % scaling factors for width and height
      \Gscale@div\@tempa\textheight{\dimexpr\ht\pandoc@box+\dp\pandoc@box\relax}%
      \Gscale@div\@tempb\linewidth{\wd\pandoc@box}%
      % select the smaller of both
      \ifdim\@tempb\p@<\@tempa\p@
        \let\@tempa\@tempb
      \fi
      % scaling accordingly (\@tempa < 1)
      \ifdim\@tempa\p@<\p@
        \scalebox{\@tempa}{\usebox\pandoc@box}%
      % scaling not needed, use as it is
      \else
        \usebox{\pandoc@box}%
      \fi
    }
    \makeatother

    % Define a nice break command that doesn't care if a line doesn't already
    % exist.
    \def\br{\hspace*{\fill} \\* }
    % Math Jax compatibility definitions
    \def\gt{>}
    \def\lt{<}
    \let\Oldtex\TeX
    \let\Oldlatex\LaTeX
    \renewcommand{\TeX}{\textrm{\Oldtex}}
    \renewcommand{\LaTeX}{\textrm{\Oldlatex}}
    % Document parameters
    % Document title
    \title{Report\_Pz\_8}
    
    
    
    
    
    
    
% Pygments definitions
\makeatletter
\def\PY@reset{\let\PY@it=\relax \let\PY@bf=\relax%
    \let\PY@ul=\relax \let\PY@tc=\relax%
    \let\PY@bc=\relax \let\PY@ff=\relax}
\def\PY@tok#1{\csname PY@tok@#1\endcsname}
\def\PY@toks#1+{\ifx\relax#1\empty\else%
    \PY@tok{#1}\expandafter\PY@toks\fi}
\def\PY@do#1{\PY@bc{\PY@tc{\PY@ul{%
    \PY@it{\PY@bf{\PY@ff{#1}}}}}}}
\def\PY#1#2{\PY@reset\PY@toks#1+\relax+\PY@do{#2}}

\@namedef{PY@tok@w}{\def\PY@tc##1{\textcolor[rgb]{0.73,0.73,0.73}{##1}}}
\@namedef{PY@tok@c}{\let\PY@it=\textit\def\PY@tc##1{\textcolor[rgb]{0.24,0.48,0.48}{##1}}}
\@namedef{PY@tok@cp}{\def\PY@tc##1{\textcolor[rgb]{0.61,0.40,0.00}{##1}}}
\@namedef{PY@tok@k}{\let\PY@bf=\textbf\def\PY@tc##1{\textcolor[rgb]{0.00,0.50,0.00}{##1}}}
\@namedef{PY@tok@kp}{\def\PY@tc##1{\textcolor[rgb]{0.00,0.50,0.00}{##1}}}
\@namedef{PY@tok@kt}{\def\PY@tc##1{\textcolor[rgb]{0.69,0.00,0.25}{##1}}}
\@namedef{PY@tok@o}{\def\PY@tc##1{\textcolor[rgb]{0.40,0.40,0.40}{##1}}}
\@namedef{PY@tok@ow}{\let\PY@bf=\textbf\def\PY@tc##1{\textcolor[rgb]{0.67,0.13,1.00}{##1}}}
\@namedef{PY@tok@nb}{\def\PY@tc##1{\textcolor[rgb]{0.00,0.50,0.00}{##1}}}
\@namedef{PY@tok@nf}{\def\PY@tc##1{\textcolor[rgb]{0.00,0.00,1.00}{##1}}}
\@namedef{PY@tok@nc}{\let\PY@bf=\textbf\def\PY@tc##1{\textcolor[rgb]{0.00,0.00,1.00}{##1}}}
\@namedef{PY@tok@nn}{\let\PY@bf=\textbf\def\PY@tc##1{\textcolor[rgb]{0.00,0.00,1.00}{##1}}}
\@namedef{PY@tok@ne}{\let\PY@bf=\textbf\def\PY@tc##1{\textcolor[rgb]{0.80,0.25,0.22}{##1}}}
\@namedef{PY@tok@nv}{\def\PY@tc##1{\textcolor[rgb]{0.10,0.09,0.49}{##1}}}
\@namedef{PY@tok@no}{\def\PY@tc##1{\textcolor[rgb]{0.53,0.00,0.00}{##1}}}
\@namedef{PY@tok@nl}{\def\PY@tc##1{\textcolor[rgb]{0.46,0.46,0.00}{##1}}}
\@namedef{PY@tok@ni}{\let\PY@bf=\textbf\def\PY@tc##1{\textcolor[rgb]{0.44,0.44,0.44}{##1}}}
\@namedef{PY@tok@na}{\def\PY@tc##1{\textcolor[rgb]{0.41,0.47,0.13}{##1}}}
\@namedef{PY@tok@nt}{\let\PY@bf=\textbf\def\PY@tc##1{\textcolor[rgb]{0.00,0.50,0.00}{##1}}}
\@namedef{PY@tok@nd}{\def\PY@tc##1{\textcolor[rgb]{0.67,0.13,1.00}{##1}}}
\@namedef{PY@tok@s}{\def\PY@tc##1{\textcolor[rgb]{0.73,0.13,0.13}{##1}}}
\@namedef{PY@tok@sd}{\let\PY@it=\textit\def\PY@tc##1{\textcolor[rgb]{0.73,0.13,0.13}{##1}}}
\@namedef{PY@tok@si}{\let\PY@bf=\textbf\def\PY@tc##1{\textcolor[rgb]{0.64,0.35,0.47}{##1}}}
\@namedef{PY@tok@se}{\let\PY@bf=\textbf\def\PY@tc##1{\textcolor[rgb]{0.67,0.36,0.12}{##1}}}
\@namedef{PY@tok@sr}{\def\PY@tc##1{\textcolor[rgb]{0.64,0.35,0.47}{##1}}}
\@namedef{PY@tok@ss}{\def\PY@tc##1{\textcolor[rgb]{0.10,0.09,0.49}{##1}}}
\@namedef{PY@tok@sx}{\def\PY@tc##1{\textcolor[rgb]{0.00,0.50,0.00}{##1}}}
\@namedef{PY@tok@m}{\def\PY@tc##1{\textcolor[rgb]{0.40,0.40,0.40}{##1}}}
\@namedef{PY@tok@gh}{\let\PY@bf=\textbf\def\PY@tc##1{\textcolor[rgb]{0.00,0.00,0.50}{##1}}}
\@namedef{PY@tok@gu}{\let\PY@bf=\textbf\def\PY@tc##1{\textcolor[rgb]{0.50,0.00,0.50}{##1}}}
\@namedef{PY@tok@gd}{\def\PY@tc##1{\textcolor[rgb]{0.63,0.00,0.00}{##1}}}
\@namedef{PY@tok@gi}{\def\PY@tc##1{\textcolor[rgb]{0.00,0.52,0.00}{##1}}}
\@namedef{PY@tok@gr}{\def\PY@tc##1{\textcolor[rgb]{0.89,0.00,0.00}{##1}}}
\@namedef{PY@tok@ge}{\let\PY@it=\textit}
\@namedef{PY@tok@gs}{\let\PY@bf=\textbf}
\@namedef{PY@tok@gp}{\let\PY@bf=\textbf\def\PY@tc##1{\textcolor[rgb]{0.00,0.00,0.50}{##1}}}
\@namedef{PY@tok@go}{\def\PY@tc##1{\textcolor[rgb]{0.44,0.44,0.44}{##1}}}
\@namedef{PY@tok@gt}{\def\PY@tc##1{\textcolor[rgb]{0.00,0.27,0.87}{##1}}}
\@namedef{PY@tok@err}{\def\PY@bc##1{{\setlength{\fboxsep}{\string -\fboxrule}\fcolorbox[rgb]{1.00,0.00,0.00}{1,1,1}{\strut ##1}}}}
\@namedef{PY@tok@kc}{\let\PY@bf=\textbf\def\PY@tc##1{\textcolor[rgb]{0.00,0.50,0.00}{##1}}}
\@namedef{PY@tok@kd}{\let\PY@bf=\textbf\def\PY@tc##1{\textcolor[rgb]{0.00,0.50,0.00}{##1}}}
\@namedef{PY@tok@kn}{\let\PY@bf=\textbf\def\PY@tc##1{\textcolor[rgb]{0.00,0.50,0.00}{##1}}}
\@namedef{PY@tok@kr}{\let\PY@bf=\textbf\def\PY@tc##1{\textcolor[rgb]{0.00,0.50,0.00}{##1}}}
\@namedef{PY@tok@bp}{\def\PY@tc##1{\textcolor[rgb]{0.00,0.50,0.00}{##1}}}
\@namedef{PY@tok@fm}{\def\PY@tc##1{\textcolor[rgb]{0.00,0.00,1.00}{##1}}}
\@namedef{PY@tok@vc}{\def\PY@tc##1{\textcolor[rgb]{0.10,0.09,0.49}{##1}}}
\@namedef{PY@tok@vg}{\def\PY@tc##1{\textcolor[rgb]{0.10,0.09,0.49}{##1}}}
\@namedef{PY@tok@vi}{\def\PY@tc##1{\textcolor[rgb]{0.10,0.09,0.49}{##1}}}
\@namedef{PY@tok@vm}{\def\PY@tc##1{\textcolor[rgb]{0.10,0.09,0.49}{##1}}}
\@namedef{PY@tok@sa}{\def\PY@tc##1{\textcolor[rgb]{0.73,0.13,0.13}{##1}}}
\@namedef{PY@tok@sb}{\def\PY@tc##1{\textcolor[rgb]{0.73,0.13,0.13}{##1}}}
\@namedef{PY@tok@sc}{\def\PY@tc##1{\textcolor[rgb]{0.73,0.13,0.13}{##1}}}
\@namedef{PY@tok@dl}{\def\PY@tc##1{\textcolor[rgb]{0.73,0.13,0.13}{##1}}}
\@namedef{PY@tok@s2}{\def\PY@tc##1{\textcolor[rgb]{0.73,0.13,0.13}{##1}}}
\@namedef{PY@tok@sh}{\def\PY@tc##1{\textcolor[rgb]{0.73,0.13,0.13}{##1}}}
\@namedef{PY@tok@s1}{\def\PY@tc##1{\textcolor[rgb]{0.73,0.13,0.13}{##1}}}
\@namedef{PY@tok@mb}{\def\PY@tc##1{\textcolor[rgb]{0.40,0.40,0.40}{##1}}}
\@namedef{PY@tok@mf}{\def\PY@tc##1{\textcolor[rgb]{0.40,0.40,0.40}{##1}}}
\@namedef{PY@tok@mh}{\def\PY@tc##1{\textcolor[rgb]{0.40,0.40,0.40}{##1}}}
\@namedef{PY@tok@mi}{\def\PY@tc##1{\textcolor[rgb]{0.40,0.40,0.40}{##1}}}
\@namedef{PY@tok@il}{\def\PY@tc##1{\textcolor[rgb]{0.40,0.40,0.40}{##1}}}
\@namedef{PY@tok@mo}{\def\PY@tc##1{\textcolor[rgb]{0.40,0.40,0.40}{##1}}}
\@namedef{PY@tok@ch}{\let\PY@it=\textit\def\PY@tc##1{\textcolor[rgb]{0.24,0.48,0.48}{##1}}}
\@namedef{PY@tok@cm}{\let\PY@it=\textit\def\PY@tc##1{\textcolor[rgb]{0.24,0.48,0.48}{##1}}}
\@namedef{PY@tok@cpf}{\let\PY@it=\textit\def\PY@tc##1{\textcolor[rgb]{0.24,0.48,0.48}{##1}}}
\@namedef{PY@tok@c1}{\let\PY@it=\textit\def\PY@tc##1{\textcolor[rgb]{0.24,0.48,0.48}{##1}}}
\@namedef{PY@tok@cs}{\let\PY@it=\textit\def\PY@tc##1{\textcolor[rgb]{0.24,0.48,0.48}{##1}}}

\def\PYZbs{\char`\\}
\def\PYZus{\char`\_}
\def\PYZob{\char`\{}
\def\PYZcb{\char`\}}
\def\PYZca{\char`\^}
\def\PYZam{\char`\&}
\def\PYZlt{\char`\<}
\def\PYZgt{\char`\>}
\def\PYZsh{\char`\#}
\def\PYZpc{\char`\%}
\def\PYZdl{\char`\$}
\def\PYZhy{\char`\-}
\def\PYZsq{\char`\'}
\def\PYZdq{\char`\"}
\def\PYZti{\char`\~}
% for compatibility with earlier versions
\def\PYZat{@}
\def\PYZlb{[}
\def\PYZrb{]}
\makeatother


    % For linebreaks inside Verbatim environment from package fancyvrb.
    \makeatletter
        \newbox\Wrappedcontinuationbox
        \newbox\Wrappedvisiblespacebox
        \newcommand*\Wrappedvisiblespace {\textcolor{red}{\textvisiblespace}}
        \newcommand*\Wrappedcontinuationsymbol {\textcolor{red}{\llap{\tiny$\m@th\hookrightarrow$}}}
        \newcommand*\Wrappedcontinuationindent {3ex }
        \newcommand*\Wrappedafterbreak {\kern\Wrappedcontinuationindent\copy\Wrappedcontinuationbox}
        % Take advantage of the already applied Pygments mark-up to insert
        % potential linebreaks for TeX processing.
        %        {, <, #, %, $, ' and ": go to next line.
        %        _, }, ^, &, >, - and ~: stay at end of broken line.
        % Use of \textquotesingle for straight quote.
        \newcommand*\Wrappedbreaksatspecials {%
            \def\PYGZus{\discretionary{\char`\_}{\Wrappedafterbreak}{\char`\_}}%
            \def\PYGZob{\discretionary{}{\Wrappedafterbreak\char`\{}{\char`\{}}%
            \def\PYGZcb{\discretionary{\char`\}}{\Wrappedafterbreak}{\char`\}}}%
            \def\PYGZca{\discretionary{\char`\^}{\Wrappedafterbreak}{\char`\^}}%
            \def\PYGZam{\discretionary{\char`\&}{\Wrappedafterbreak}{\char`\&}}%
            \def\PYGZlt{\discretionary{}{\Wrappedafterbreak\char`\<}{\char`\<}}%
            \def\PYGZgt{\discretionary{\char`\>}{\Wrappedafterbreak}{\char`\>}}%
            \def\PYGZsh{\discretionary{}{\Wrappedafterbreak\char`\#}{\char`\#}}%
            \def\PYGZpc{\discretionary{}{\Wrappedafterbreak\char`\%}{\char`\%}}%
            \def\PYGZdl{\discretionary{}{\Wrappedafterbreak\char`\$}{\char`\$}}%
            \def\PYGZhy{\discretionary{\char`\-}{\Wrappedafterbreak}{\char`\-}}%
            \def\PYGZsq{\discretionary{}{\Wrappedafterbreak\textquotesingle}{\textquotesingle}}%
            \def\PYGZdq{\discretionary{}{\Wrappedafterbreak\char`\"}{\char`\"}}%
            \def\PYGZti{\discretionary{\char`\~}{\Wrappedafterbreak}{\char`\~}}%
        }
        % Some characters . , ; ? ! / are not pygmentized.
        % This macro makes them "active" and they will insert potential linebreaks
        \newcommand*\Wrappedbreaksatpunct {%
            \lccode`\~`\.\lowercase{\def~}{\discretionary{\hbox{\char`\.}}{\Wrappedafterbreak}{\hbox{\char`\.}}}%
            \lccode`\~`\,\lowercase{\def~}{\discretionary{\hbox{\char`\,}}{\Wrappedafterbreak}{\hbox{\char`\,}}}%
            \lccode`\~`\;\lowercase{\def~}{\discretionary{\hbox{\char`\;}}{\Wrappedafterbreak}{\hbox{\char`\;}}}%
            \lccode`\~`\:\lowercase{\def~}{\discretionary{\hbox{\char`\:}}{\Wrappedafterbreak}{\hbox{\char`\:}}}%
            \lccode`\~`\?\lowercase{\def~}{\discretionary{\hbox{\char`\?}}{\Wrappedafterbreak}{\hbox{\char`\?}}}%
            \lccode`\~`\!\lowercase{\def~}{\discretionary{\hbox{\char`\!}}{\Wrappedafterbreak}{\hbox{\char`\!}}}%
            \lccode`\~`\/\lowercase{\def~}{\discretionary{\hbox{\char`\/}}{\Wrappedafterbreak}{\hbox{\char`\/}}}%
            \catcode`\.\active
            \catcode`\,\active
            \catcode`\;\active
            \catcode`\:\active
            \catcode`\?\active
            \catcode`\!\active
            \catcode`\/\active
            \lccode`\~`\~
        }
    \makeatother

    \let\OriginalVerbatim=\Verbatim
    \makeatletter
    \renewcommand{\Verbatim}[1][1]{%
        %\parskip\z@skip
        \sbox\Wrappedcontinuationbox {\Wrappedcontinuationsymbol}%
        \sbox\Wrappedvisiblespacebox {\FV@SetupFont\Wrappedvisiblespace}%
        \def\FancyVerbFormatLine ##1{\hsize\linewidth
            \vtop{\raggedright\hyphenpenalty\z@\exhyphenpenalty\z@
                \doublehyphendemerits\z@\finalhyphendemerits\z@
                \strut ##1\strut}%
        }%
        % If the linebreak is at a space, the latter will be displayed as visible
        % space at end of first line, and a continuation symbol starts next line.
        % Stretch/shrink are however usually zero for typewriter font.
        \def\FV@Space {%
            \nobreak\hskip\z@ plus\fontdimen3\font minus\fontdimen4\font
            \discretionary{\copy\Wrappedvisiblespacebox}{\Wrappedafterbreak}
            {\kern\fontdimen2\font}%
        }%

        % Allow breaks at special characters using \PYG... macros.
        \Wrappedbreaksatspecials
        % Breaks at punctuation characters . , ; ? ! and / need catcode=\active
        \OriginalVerbatim[#1,codes*=\Wrappedbreaksatpunct]%
    }
    \makeatother

    % Exact colors from NB
    \definecolor{incolor}{HTML}{303F9F}
    \definecolor{outcolor}{HTML}{D84315}
    \definecolor{cellborder}{HTML}{CFCFCF}
    \definecolor{cellbackground}{HTML}{F7F7F7}

    % prompt
    \makeatletter
    \newcommand{\boxspacing}{\kern\kvtcb@left@rule\kern\kvtcb@boxsep}
    \makeatother
    \newcommand{\prompt}[4]{
        {\ttfamily\llap{{\color{#2}[#3]:\hspace{3pt}#4}}\vspace{-\baselineskip}}
    }
    

    
    % Prevent overflowing lines due to hard-to-break entities
    \sloppy
    % Setup hyperref package
    \hypersetup{
      breaklinks=true,  % so long urls are correctly broken across lines
      colorlinks=true,
      urlcolor=urlcolor,
      linkcolor=linkcolor,
      citecolor=citecolor,
      }
    % Slightly bigger margins than the latex defaults
    
    \geometry{verbose,tmargin=1in,bmargin=1in,lmargin=1in,rmargin=1in}
    
    

\begin{document}
    
    \maketitle
    
    

    
    \section{Практична робота
№8}\label{ux43fux440ux430ux43aux442ux438ux447ux43dux430-ux440ux43eux431ux43eux442ux430-8}

\textbf{Виконав} студент групи КН-24-1 Озівський В.В.

\textbf{Тема:} Основи вибіркового методу.

\textbf{Мета:} набути практичних навичок у розв'язанні типових задач з
основ вибіркового методу, точкового та інтервального оцінювання числових
характеристик випадкової величини

\section{Хід
роботи}\label{ux445ux456ux434-ux440ux43eux431ux43eux442ux438}

    \textbf{Завдання 18.}

\textbf{Виконання індивідуального завдання згідно з прикладами 4.1-4.20}

\textbf{Вихідна вибірка:} \(X = [3, 4, 3, 5, 10]\), об'єм вибірки
\(n=5\).

    \paragraph{\texorpdfstring{\textbf{1. Побудова варіаційного ряду (за
аналогією з Прикладом
4.3)}}{1. Побудова варіаційного ряду (за аналогією з Прикладом 4.3)}}\label{ux43fux43eux431ux443ux434ux43eux432ux430-ux432ux430ux440ux456ux430ux446ux456ux439ux43dux43eux433ux43e-ux440ux44fux434ux443-ux437ux430-ux430ux43dux430ux43bux43eux433ux456ux454ux44e-ux437-ux43fux440ux438ux43aux43bux430ux434ux43eux43c-4.3}

Відсортована за зростанням вибірка (варіаційний ряд):
\[ [3, 3, 4, 5, 10] \] Унікальні варіанти (\(\alpha\)):
\[ [3, 4, 5, 10] \]

    \paragraph{\texorpdfstring{\textbf{2. Побудова емпіричного закону
розподілу (за аналогією з Прикладом
4.4)}}{2. Побудова емпіричного закону розподілу (за аналогією з Прикладом 4.4)}}\label{ux43fux43eux431ux443ux434ux43eux432ux430-ux435ux43cux43fux456ux440ux438ux447ux43dux43eux433ux43e-ux437ux430ux43aux43eux43dux443-ux440ux43eux437ux43fux43eux434ux456ux43bux443-ux437ux430-ux430ux43dux430ux43bux43eux433ux456ux454ux44e-ux437-ux43fux440ux438ux43aux43bux430ux434ux43eux43c-4.4}

Складаємо таблицю відносних частот (\(\omega_i = m_i/n\)):

\begin{longtable}[]{@{}ll@{}}
\toprule\noalign{}
Варіанта (\(x_i\)) & Відносна частота (\(\omega_i\)) \\
\midrule\noalign{}
\endhead
\bottomrule\noalign{}
\endlastfoot
3 & 2/5 = 0.4 \\
4 & 1/5 = 0.2 \\
5 & 1/5 = 0.2 \\
10 & 1/5 = 0.2 \\
\end{longtable}

    \paragraph{\texorpdfstring{\textbf{3. Побудова гістограми (за аналогією
з Прикладом
4.5)}}{3. Побудова гістограми (за аналогією з Прикладом 4.5)}}\label{ux43fux43eux431ux443ux434ux43eux432ux430-ux433ux456ux441ux442ux43eux433ux440ux430ux43cux438-ux437ux430-ux430ux43dux430ux43bux43eux433ux456ux454ux44e-ux437-ux43fux440ux438ux43aux43bux430ux434ux43eux43c-4.5}

\begin{itemize}
\tightlist
\item
  \textbf{Кількість інтервалів (формула Стерджесса):}
  \(k = 1 + 1.332 \cdot \ln(5) \approx 3.143 \approx 3\).
\item
  \textbf{Ширина інтервалу:}
  \(h = (X_{max} - X_{min}) / k = (10 - 3) / 3 \approx 2.33\).
\item
  \textbf{Інтервальна таблиця частот:}
\end{itemize}

\begin{longtable}[]{@{}ll@{}}
\toprule\noalign{}
Інтервал & Частота (\(\omega_i\)) \\
\midrule\noalign{}
\endhead
\bottomrule\noalign{}
\endlastfoot
{[}3; 5.33) & 4/5 = 0.8 \\
{[}5.33; 7.66) & 0/5 = 0 \\
{[}7.66; 10{]} & 1/5 = 0.2 \\
\end{longtable}

    \paragraph{\texorpdfstring{\textbf{4. Побудова емпіричної функції
розподілу (за аналогією з Прикладом
4.6)}}{4. Побудова емпіричної функції розподілу (за аналогією з Прикладом 4.6)}}\label{ux43fux43eux431ux443ux434ux43eux432ux430-ux435ux43cux43fux456ux440ux438ux447ux43dux43eux457-ux444ux443ux43dux43aux446ux456ux457-ux440ux43eux437ux43fux43eux434ux456ux43bux443-ux437ux430-ux430ux43dux430ux43bux43eux433ux456ux454ux44e-ux437-ux43fux440ux438ux43aux43bux430ux434ux43eux43c-4.6}

\[ F_n^*(x) = \begin{cases} 0, & x \le 3 \\ 0.4, & 3 < x \le 4 \\ 0.6, & 4 < x \le 5 \\ 0.8, & 5 < x \le 10 \\ 1, & x > 10 \end{cases} \]

    \paragraph{\texorpdfstring{\textbf{5. Розрахунок мір центральної
тенденції (Приклади
4.7-4.9)}}{5. Розрахунок мір центральної тенденції (Приклади 4.7-4.9)}}\label{ux440ux43eux437ux440ux430ux445ux443ux43dux43eux43a-ux43cux456ux440-ux446ux435ux43dux442ux440ux430ux43bux44cux43dux43eux457-ux442ux435ux43dux434ux435ux43dux446ux456ux457-ux43fux440ux438ux43aux43bux430ux434ux438-4.7-4.9}

\begin{itemize}
\tightlist
\item
  \textbf{Медіана (Me):} \(n=5\) (непарне),
  \(Me = \alpha_{(5+1)/2} = \alpha_3 = 4\).
\item
  \textbf{Середнє арифметичне (\(\bar{x}\)):}
  \(\bar{x} = \frac{3+4+3+5+10}{5} = 5\).
\item
  \textbf{Мода (Mo):} Найчастіше значення --- 3. \(Mo = 3\).
\end{itemize}

    \paragraph{\texorpdfstring{\textbf{6. Розрахунок мір розсіювання
(Приклади
4.10-4.13)}}{6. Розрахунок мір розсіювання (Приклади 4.10-4.13)}}\label{ux440ux43eux437ux440ux430ux445ux443ux43dux43eux43a-ux43cux456ux440-ux440ux43eux437ux441ux456ux44eux432ux430ux43dux43dux44f-ux43fux440ux438ux43aux43bux430ux434ux438-4.10-4.13}

\begin{itemize}
\tightlist
\item
  \textbf{Розмах (R):} \(R = 10 - 3 = 7\).
\item
  \textbf{Виправлена вибіркова дисперсія (\(s^2\)):}
  \(s^2 = \frac{1}{4} [ (3-5)^2 + (4-5)^2 + (3-5)^2 + (5-5)^2 + (10-5)^2 ] = \frac{34}{4} = 8.5\).
\item
  \textbf{Виправлене СКВ (\(s\)):} \(s = \sqrt{8.5} \approx 2.915\).
\item
  \textbf{Середня абсолютна похибка (MAE):}
  \(MAE = \frac{1}{5} [|3-5| + |4-5| + |3-5| + |5-5| + |10-5|] = \frac{1}{5}[2+1+2+0+5] = \frac{10}{5} = 2\).
\end{itemize}

    \paragraph{\texorpdfstring{\textbf{7. Розрахунок мір форми розподілу
(Приклади
4.14-4.17)}}{7. Розрахунок мір форми розподілу (Приклади 4.14-4.17)}}\label{ux440ux43eux437ux440ux430ux445ux443ux43dux43eux43a-ux43cux456ux440-ux444ux43eux440ux43cux438-ux440ux43eux437ux43fux43eux434ux456ux43bux443-ux43fux440ux438ux43aux43bux430ux434ux438-4.14-4.17}

\begin{itemize}
\tightlist
\item
  \textbf{Вибіркова асиметрія (\(\tilde{A}_s\)):}
  \(\tilde{A}_s = \frac{5}{(4)(3)(2.915)^3} \sum (x_i - 5)^3 = \frac{5}{12 \cdot 24.76} \cdot 108 \approx 1.817\).
\item
  \textbf{Стандартизована асиметрія (\(z_1\)):}
  \(z_1 = \frac{1.817}{\sqrt{6/5}} \approx 1.659\).
\item
  \textbf{Вибірковий ексцес (\(\tilde{E}_k\)):}
  \(\tilde{E}_k = \frac{5(6)}{(4)(3)(2)(8.5)^2} \sum (x_i - 5)^4 - \frac{3(4)^2}{(3)(2)} = \frac{30}{1734} \cdot 658 - 8 \approx 11.419 - 8 = 3.419\).
\item
  \textbf{Стандартизований ексцес (\(z_2\)):}
  \(z_2 = \frac{3.419}{\sqrt{24/5}} \approx 1.561\).
\end{itemize}

    \paragraph{\texorpdfstring{\textbf{8. Розрахунок інтервальних оцінок
(Приклади
4.19-4.20)}}{8. Розрахунок інтервальних оцінок (Приклади 4.19-4.20)}}\label{ux440ux43eux437ux440ux430ux445ux443ux43dux43eux43a-ux456ux43dux442ux435ux440ux432ux430ux43bux44cux43dux438ux445-ux43eux446ux456ux43dux43eux43a-ux43fux440ux438ux43aux43bux430ux434ux438-4.19-4.20}

\begin{itemize}
\tightlist
\item
  \textbf{Інтервальна оцінка для середнього (\(a\)) з надійністю
  \(\gamma=0.95\):} Для \(k=4\), \(t_{табл} \approx 2.776\).
  \[ 5 \pm \frac{2.915}{\sqrt{5}} \cdot 2.776 \implies 5 \pm 3.619 \]
  \textbf{Довірчий інтервал: (1.381, 8.619)}.
\item
  \textbf{Інтервальна оцінка для дисперсії (\(\sigma^2\)) з надійністю
  \(\gamma=0.95\):} Для \(k=4\), \(\chi^2_{0.975,4} \approx 11.143\) та
  \(\chi^2_{0.025,4} \approx 0.484\).
  \[ \frac{4 \cdot 8.5}{11.143} < \sigma^2 < \frac{4 \cdot 8.5}{0.484} \implies 3.051 < \sigma^2 < 70.248 \]
  \textbf{Довірчий інтервал для \(\sigma\): (1.747, 8.381)}.
\end{itemize}

    \subsubsection{\texorpdfstring{\textbf{9. П'ятиквантильний графік
(``Ящик з
вусами'')}}{9. П'ятиквантильний графік (``Ящик з вусами'')}}\label{ux43fux44fux442ux438ux43aux432ux430ux43dux442ux438ux43bux44cux43dux438ux439-ux433ux440ux430ux444ux456ux43a-ux44fux449ux438ux43a-ux437-ux432ux443ux441ux430ux43cux438}

\begin{itemize}
\tightlist
\item
  \textbf{Медіана (Q2):} 4.
\item
  \textbf{Перший квартиль (Q1):} 3.
\item
  \textbf{Третій квартиль (Q3):} 7.5.
\item
  \textbf{Міжквартильний розмах (IQR):} \(7.5 - 3 = 4.5\).
\item
  \textbf{Межі для викидів:}

  \begin{itemize}
  \tightlist
  \item
    Нижня: \(Q1 - 1.5 \cdot IQR = 3 - 6.75 = -3.75\).
  \item
    Верхня: \(Q3 + 1.5 \cdot IQR = 7.5 + 6.75 = 14.25\).
  \end{itemize}
\item
  \textbf{Висновок:} Всі значення вибірки знаходяться в межах
  \texttt{{[}-3.75,\ 14.25{]}}, тому викидів немає. ``Вуса'' графіка
  будуть відповідати мінімальному (3) та максимальному (10) значенням.
\end{itemize}

    \subsubsection{Контрольні
питання}\label{ux43aux43eux43dux442ux440ux43eux43bux44cux43dux456-ux43fux438ux442ux430ux43dux43dux44f}

\paragraph{1. Що таке вибірковий метод і як він використовується в
статистиці?}\label{ux449ux43e-ux442ux430ux43aux435-ux432ux438ux431ux456ux440ux43aux43eux432ux438ux439-ux43cux435ux442ux43eux434-ux456-ux44fux43a-ux432ux456ux43d-ux432ux438ux43aux43eux440ux438ux441ux442ux43eux432ux443ux454ux442ux44cux441ux44f-ux432-ux441ux442ux430ux442ux438ux441ux442ux438ux446ux456}

Це метод дослідження, при якому висновки про всю сукупність (генеральну
сукупність) робляться на основі аналізу її невеликої частини (вибірки).
Він використовується для економії часу та ресурсів, коли дослідити всю
сукупність неможливо або недоцільно.

\paragraph{2. Які є основні точкові статистичні оцінки, і як вони
обчислюються?}\label{ux44fux43aux456-ux454-ux43eux441ux43dux43eux432ux43dux456-ux442ux43eux447ux43aux43eux432ux456-ux441ux442ux430ux442ux438ux441ux442ux438ux447ux43dux456-ux43eux446ux456ux43dux43aux438-ux456-ux44fux43a-ux432ux43eux43dux438-ux43eux431ux447ux438ux441ux43bux44eux44eux442ux44cux441ux44f}

\begin{itemize}
\tightlist
\item
  \textbf{Вибіркове середнє (\(\bar{x}\)):} Сума всіх значень, поділена
  на їх кількість. Оцінює середнє генеральної сукупності.
\item
  \textbf{Вибіркова дисперсія (\(s^2\)):} Середній квадрат відхилень
  значень від вибіркового середнього. Оцінює розкид даних.
\item
  \textbf{Вибіркова частка (\(\omega\)):} Відношення кількості елементів
  з певною ознакою до загального обсягу вибірки.
\end{itemize}

\paragraph{3. Які фактори впливають на точність статистичних
оцінок?}\label{ux44fux43aux456-ux444ux430ux43aux442ux43eux440ux438-ux432ux43fux43bux438ux432ux430ux44eux442ux44c-ux43dux430-ux442ux43eux447ux43dux456ux441ux442ux44c-ux441ux442ux430ux442ux438ux441ux442ux438ux447ux43dux438ux445-ux43eux446ux456ux43dux43eux43a}

\begin{itemize}
\tightlist
\item
  \textbf{Обсяг вибірки:} Чим більша вибірка, тим вища точність.
\item
  \textbf{Варіативність (мінливість) даних:} Чим більший розкид даних у
  генеральній сукупності, тим менша точність.
\item
  \textbf{Метод формування вибірки:} Репрезентативна (випадкова) вибірка
  дає більш точні результати, ніж зміщена.
\end{itemize}

\paragraph{4. Як визначається вибіркове середнє і вибіркова
дисперсія?}\label{ux44fux43a-ux432ux438ux437ux43dux430ux447ux430ux454ux442ux44cux441ux44f-ux432ux438ux431ux456ux440ux43aux43eux432ux435-ux441ux435ux440ux435ux434ux43dux454-ux456-ux432ux438ux431ux456ux440ux43aux43eux432ux430-ux434ux438ux441ux43fux435ux440ux441ux456ux44f}

\begin{itemize}
\tightlist
\item
  \textbf{Вибіркове середнє (\(\bar{x}\)):}
  \[ \bar{x} = \frac{1}{n} \sum_{i=1}^{n} x_i \]
\item
  \textbf{Виправлена вибіркова дисперсія (\(s^2\)):}
  \[ s^2 = \frac{1}{n-1} \sum_{i=1}^{n} (x_i - \bar{x})^2 \]
\end{itemize}

\paragraph{5. Що таке точні вибіркові розподілення і як вони допомагають
у роботі з вибірковими
оцінками?}\label{ux449ux43e-ux442ux430ux43aux435-ux442ux43eux447ux43dux456-ux432ux438ux431ux456ux440ux43aux43eux432ux456-ux440ux43eux437ux43fux43eux434ux456ux43bux435ux43dux43dux44f-ux456-ux44fux43a-ux432ux43eux43dux438-ux434ux43eux43fux43eux43cux430ux433ux430ux44eux442ux44c-ux443-ux440ux43eux431ux43eux442ux456-ux437-ux432ux438ux431ux456ux440ux43aux43eux432ux438ux43cux438-ux43eux446ux456ux43dux43aux430ux43cux438}

Це теоретичні розподіли ймовірностей для статистик, розрахованих за
вибірками (наприклад, розподіл Стьюдента, \(\chi^2\), Фішера). Вони
дозволяють будувати довірчі інтервали та перевіряти статистичні
гіпотези, не знаючи точних параметрів генеральної сукупності.

\paragraph{6. Які властивості мають інтервальні статистичні
оцінки?}\label{ux44fux43aux456-ux432ux43bux430ux441ux442ux438ux432ux43eux441ux442ux456-ux43cux430ux44eux442ux44c-ux456ux43dux442ux435ux440ux432ux430ux43bux44cux43dux456-ux441ux442ux430ux442ux438ux441ux442ux438ux447ux43dux456-ux43eux446ux456ux43dux43aux438}

\begin{itemize}
\tightlist
\item
  \textbf{Надійність (\(\gamma\)):} Ймовірність того, що інтервал накриє
  істинне значення параметра. Зазвичай приймається 95\% або 99\%.
\item
  \textbf{Точність:} Ширина довірчого інтервалу. Чим вужчий інтервал,
  тим точніша оцінка. Інтервальна оцінка, на відміну від точкової,
  враховує похибку вимірювання.
\end{itemize}

\paragraph{7. Як будується довірчий інтервал для параметра генеральної
сукупності?}\label{ux44fux43a-ux431ux443ux434ux443ux454ux442ux44cux441ux44f-ux434ux43eux432ux456ux440ux447ux438ux439-ux456ux43dux442ux435ux440ux432ux430ux43b-ux434ux43bux44f-ux43fux430ux440ux430ux43cux435ux442ux440ux430-ux433ux435ux43dux435ux440ux430ux43bux44cux43dux43eux457-ux441ux443ux43aux443ux43fux43dux43eux441ux442ux456}

Він будується навколо точкової оцінки параметра. Межі інтервалу
визначаються за формулою:
\[ \text{Точкова оцінка} \pm (\text{Квантиль розподілу} \times \text{Стандартна похибка оцінки}) \]
Квантиль залежить від обраної надійності та відповідного розподілу
(Стьюдента, нормального тощо).

\paragraph{8. Як визначається довірчий інтервал для середнього значення
генеральної
сукупності?}\label{ux44fux43a-ux432ux438ux437ux43dux430ux447ux430ux454ux442ux44cux441ux44f-ux434ux43eux432ux456ux440ux447ux438ux439-ux456ux43dux442ux435ux440ux432ux430ux43b-ux434ux43bux44f-ux441ux435ux440ux435ux434ux43dux44cux43eux433ux43e-ux437ux43dux430ux447ux435ux43dux43dux44f-ux433ux435ux43dux435ux440ux430ux43bux44cux43dux43eux457-ux441ux443ux43aux443ux43fux43dux43eux441ux442ux456}

Якщо дисперсія генеральної сукупності невідома (найчастіший випадок),
інтервал визначається за формулою:
\[ \bar{x} - t \cdot \frac{s}{\sqrt{n}} < a < \bar{x} + t \cdot \frac{s}{\sqrt{n}} \]
де \(\bar{x}\) --- вибіркове середнє, \(s\) --- вибіркове СКВ, \(n\) ---
обсяг вибірки, \(t\) --- квантиль розподілу Стьюдента.

\paragraph{9. Як використовувати інтервальні статистичні оцінки для
прийняття
рішень?}\label{ux44fux43a-ux432ux438ux43aux43eux440ux438ux441ux442ux43eux432ux443ux432ux430ux442ux438-ux456ux43dux442ux435ux440ux432ux430ux43bux44cux43dux456-ux441ux442ux430ux442ux438ux441ux442ux438ux447ux43dux456-ux43eux446ux456ux43dux43aux438-ux434ux43bux44f-ux43fux440ux438ux439ux43dux44fux442ux442ux44f-ux440ux456ux448ux435ux43dux44c}

Інтервальні оцінки дозволяють перевірити, чи потрапляє певне гіпотетичне
значення (наприклад, норматив, плановий показник) у довірчий інтервал.
Якщо значення не потрапляє в інтервал, гіпотезу про його істинність
відхиляють. Також їх використовують для порівняння параметрів двох
різних сукупностей.

\paragraph{10. Які методи можна використовувати для визначення обсягу
вибірки для отримання точних статистичних
оцінок?}\label{ux44fux43aux456-ux43cux435ux442ux43eux434ux438-ux43cux43eux436ux43dux430-ux432ux438ux43aux43eux440ux438ux441ux442ux43eux432ux443ux432ux430ux442ux438-ux434ux43bux44f-ux432ux438ux437ux43dux430ux447ux435ux43dux43dux44f-ux43eux431ux441ux44fux433ux443-ux432ux438ux431ux456ux440ux43aux438-ux434ux43bux44f-ux43eux442ux440ux438ux43cux430ux43dux43dux44f-ux442ux43eux447ux43dux438ux445-ux441ux442ux430ux442ux438ux441ux442ux438ux447ux43dux438ux445-ux43eux446ux456ux43dux43eux43a}

Обсяг вибірки розраховується на основі трьох параметрів: 1.
\textbf{Бажана точність} (максимально допустима похибка). 2.
\textbf{Необхідна надійність} (наприклад, 95\%). 3. \textbf{Очікувана
варіативність даних} (оцінка дисперсії з попередніх досліджень або
пілотної вибірки). Існують спеціальні формули для розрахунку \(n\) для
середнього значення та для частки.

    \subsubsection{Висновок}\label{ux432ux438ux441ux43dux43eux432ux43eux43a}

Аналіз вибірки \(X =\) показав, що дані мають значний розкид
(\(s \approx 2.915\)) та правосторонню асиметрію
(\(\tilde{A}_s \approx 1.817\)), спричинену екстремальним значенням 10.
Середнє значення (\(\bar{x}=5\)) виявилося більшим за медіану
(\(Me=4\)). Незважаючи на відхилення форми розподілу від нормальної, для
такого малого об'єму вибірки ці відхилення не є статистично значущими.

    


    % Add a bibliography block to the postdoc
    
    
    
\end{document}
